% Created 2012-02-21 Tue 17:09
\documentclass[mathserif]{beamer}
\usepackage[utf8x]{inputenc}
\usepackage[T1]{fontenc}
\usepackage{fixltx2e}
\usepackage{graphicx}
\usepackage{longtable}
\usepackage{float}
\usepackage{wrapfig}
\usepackage{soul}
\usepackage{textcomp}
\usepackage{marvosym}
\usepackage{wasysym}
\usepackage{latexsym}
\usepackage{amssymb}
\usepackage{hyperref}
\tolerance=1000
\usepackage{color}
\newcommand{\colquo}[1]{\textcolor{red}{#1}}
\newcommand{\shl}[1]{\text{«}^{#1}}
\newcommand{\shr}[1]{\text{»}^{#1}}
\providecommand{\alert}[1]{\textbf{#1}}

\title{Gödel's Incompleteness Theorem}
\author{Phil Hazelden}
\date{2012-02-08}
\hypersetup{
  pdfkeywords={},
  pdfsubject={},
  pdfcreator={Emacs Org-mode version 7.8.03}}

\begin{document}

\maketitle



\section{Introduction}
\label{sec-1}
\begin{frame}
\frametitle{Overview}
\label{sec-1-1}
\begin{itemize}

\item Godel's theorem says, roughly, that there are true statements about the natural numbers which cannot be proved.\\
\label{sec-1-1-1}%
\item We prove this theorem by constructing such a statement.\\
\label{sec-1-1-2}%
\item Roughly speaking, ``this statement is unprovable''.
\label{sec-1-1-3}%
\begin{itemize}

\item If false, it can be proved to be true.\\
\label{sec-1-1-3-1}%
\item Therefore true, therefore unprovable.\\
\label{sec-1-1-3-2}%
\end{itemize} % ends low level

\item How is this not a proof?\\
\label{sec-1-1-4}%
\item How is this statement about the natural numbers?\\
\label{sec-1-1-5}%
\end{itemize} % ends low level
\end{frame}
\section{Peano arithmetic as a formal system}
\label{sec-2}
\begin{frame}
\frametitle{The ``MIU'' game}
\label{sec-2-1}
\begin{itemize}

\item Played with strings containing the symbols \texttt{M}, \texttt{I}, \texttt{U}.\\
\label{sec-2-1-1}%
\item Some strings are theorems. Which?\\
\label{sec-2-1-2}%
\item Derivations: making new theorems from old.
\label{sec-2-1-3}%
\begin{itemize}

\item \texttt{M$x$I} → \texttt{M$x$IU}\\
\label{sec-2-1-3-1}%
\item \texttt{M$x$} → \texttt{M$xx$}\\
\label{sec-2-1-3-2}%
\item \texttt{M$x$III$y$} → \texttt{M$x$U$y$}\\
\label{sec-2-1-3-3}%
\item \texttt{M$x$UU$y$} → \texttt{M$xy$}\\
\label{sec-2-1-3-4}%
\end{itemize} % ends low level

\item We start with a list of initial theorems (``axioms'')
\label{sec-2-1-4}%
\begin{itemize}

\item The only axiom is \texttt{MI}.\\
\label{sec-2-1-4-1}%
\end{itemize} % ends low level
\end{itemize} % ends low level
\end{frame}
\begin{frame}
\frametitle{An example derivation}
\label{sec-2-2}

\begin{itemize}
\item \texttt{MI} \hfill axiom
\item \texttt{MII} \hfill \texttt{M$x$} → \texttt{M$xx$}
\item \texttt{MIIII} \hfill \texttt{M$x$} → \texttt{M$xx$}
\item \texttt{MUI} \hfill \texttt{M$x$III$y$} → \texttt{M$x$U$y$}
\item \texttt{MUIU} \hfill \texttt{M$x$I} → \texttt{M$x$IU}
\item \texttt{MUIUUIU} \hfill \texttt{M$x$} → \texttt{M$xx$}
\item \texttt{MUIIU} \hfill \texttt{M$x$UU$y$} → \texttt{M$xy$}
\end{itemize}
\end{frame}
\begin{frame}
\frametitle{What about nontheorems?}
\label{sec-2-3}
\begin{itemize}

\item Is \texttt{MU} a theorem?\\
\label{sec-2-3-1}%
\item If it is, we can prove it by producing a derivation.
\label{sec-2-3-2}%
\begin{itemize}

\item We could search derivations methodically, and eventually find MU.\\
\label{sec-2-3-2-1}%
\end{itemize} % ends low level

\item If not, how can we prove that?
\label{sec-2-3-3}%
\begin{itemize}

\item If we search for it, we'll never find it. But we'll never see that we'll never find it.\\
\label{sec-2-3-3-1}%
\item Answer: count the number of \texttt{I}s in a theorem, modulo 3.\\
\label{sec-2-3-3-2}%
\item This is not generalisable.\\
\label{sec-2-3-3-3}%
\end{itemize} % ends low level
\end{itemize} % ends low level
\end{frame}
\begin{frame}
\frametitle{The ``Number theory'' (NT) Game}
\label{sec-2-4}
\begin{itemize}

\item A more complicated game that mathematicians like to play.\\
\label{sec-2-4-1}%
\item Symbols: 0-9, +, ×, =, ∧, ∨, ¬, ⇒, (, ), a, b, c, $\prime$, ∀, ∃\\
\label{sec-2-4-2}%
\item Some derivation rules:
\label{sec-2-4-3}%
\begin{itemize}

\item $w(x ∧ y)z$ → $w¬(¬x ∨ ¬y)z$ \hfill de Morgan's law\\
\label{sec-2-4-3-1}%
\item $(x = y)$ → $(x+1 = y+1)$ \hfill Successorship\\
\label{sec-2-4-3-2}%
\item $x$, $y$ → $(x ∧ y)$ \hfill Joining\\
\label{sec-2-4-3-3}%
\item $x\{0\}$, $∀a (x\{a\} ⇒ x\{a+1\})$ → $∀a (x\{a\})$ \hfill Induction\\
\label{sec-2-4-3-4}%
\end{itemize} % ends low level

\item Axioms:
\label{sec-2-4-4}%
\begin{itemize}

\item $∀a ¬(a+1 = 0)$\\
\label{sec-2-4-4-1}%
\item $∀a (a+0 = a)$\\
\label{sec-2-4-4-2}%
\item $∀a∀b (a + (b+1) = (a+b) + 1)$\\
\label{sec-2-4-4-3}%
\item $∀a (a×0 = 0)$\\
\label{sec-2-4-4-4}%
\item $∀a∀b (a × (b+1) = (a×b) + a)$\\
\label{sec-2-4-4-5}%
\end{itemize} % ends low level
\end{itemize} % ends low level
\end{frame}
\begin{frame}
\frametitle{Is number theory complete?}
\label{sec-2-5}
\begin{itemize}

\item A well-formed statement in the numer theory game is either:
\label{sec-2-5-1}%
\begin{itemize}

\item True or false;\\
\label{sec-2-5-1-1}%
\item A theorem or nontheorem.\\
\label{sec-2-5-1-2}%
\item If a nontheorem, its negation might be a theorem.\\
\label{sec-2-5-1-3}%
\end{itemize} % ends low level

\item We hope:
\label{sec-2-5-2}%
\begin{itemize}

\item A statement is a theorem iff it is true;\\
\label{sec-2-5-2-1}%
\item $x$ is a theorem iff $¬x$ is a nontheorem.\\
\label{sec-2-5-2-2}%
\item These are (kind of) equivalent.\\
\label{sec-2-5-2-3}%
\end{itemize} % ends low level

\item But incompleteness ruins this hope.\\
\label{sec-2-5-3}%
\end{itemize} % ends low level
\end{frame}
\section{Gödel numbering}
\label{sec-3}
\begin{frame}
\frametitle{Gödel numbering (MIU)}
\label{sec-3-1}
\begin{itemize}

\item Turn MIU-strings into numbers.\\
\label{sec-3-1-1}%
\item M → 1, I → 2, U → 3\\
\label{sec-3-1-2}%
\item New rules, where $x\shl{n}$ means $x\cdot 10^n$:
\label{sec-3-1-3}%
\begin{itemize}

\item $1\shl{n+1} + x\shl{1} + 2$ → $1\shl{n+2} + x\shl{2} + 23$ where $x < 1\shl{n}$\\
\label{sec-3-1-3-1}%
\item $1\shl{n} + x → 1\shl{2n} + x\shl{n} + x$, where $x < 1\shl{n}$\\
\label{sec-3-1-3-2}%
\item $1\shl{m+n+3} + x\shl{m+3} + 222\shl{m} + y$ → $1\shl{m+n+1} + x\shl{m+1} + 3\shl{m} + y$, where $x < 1\shl{n}$ and $y < 1\shl{m}$\\
\label{sec-3-1-3-3}%
\item $1\shl{m+n+2} + x\shl{m+2} + 33\shl{m} + y$ → $1\shl{m+n} + x\shl{m} + y$, where $x < 1\shl{n}$ and $y < 1\shl{m}$\\
\label{sec-3-1-3-4}%
\item $12$ is an axiom\\
\label{sec-3-1-3-5}%
\end{itemize} % ends low level

\item So now we have derivations like:
\label{sec-3-1-4}%
\begin{itemize}

\item 12 → 122 → 12222 → 132 → 1323 → 1323323 → 13223\\
\label{sec-3-1-4-1}%
\end{itemize} % ends low level
\end{itemize} % ends low level
\end{frame}
\begin{frame}
\frametitle{Gödel numbering (number theory)}
\label{sec-3-2}
\begin{itemize}

\item Can do the same thing to the number theory game, it's just a lot more complicated.\\
\label{sec-3-2-1}%
\item For example, De Morgan's law: $w(x ∧ y)z$ → $w¬(¬x ∨ ¬y)z$.\\
\label{sec-3-2-2}%
\small{\[
w\shl{l+m+n+3} + \colquo{\left(\right.}\shl{l+m+n+2} + x\shl{m+n+2} + \colquo{∧}\shl{m+n+1} + y\shl{n+1} + \colquo{\left.\right)}\shl{n} + z
\]
→
\[
w\shl{l+m+n+6} + \colquo{¬\left(\right.¬}\shl{l+m+n+3} + x\shl{m+n+3} + \colquo{∨¬}\shl{m+n+1} + y\shl{n+1} + \colquo{\left.\right)}\shl{n}+z
\]}
where $z < 1\shl{n}$, $y < 1\shl{m}$, $x < 1\shl{l}$.
\end{itemize} % ends low level
\end{frame}
\section{Representation}
\label{sec-4}
\begin{frame}
\frametitle{What good is this?}
\label{sec-4-1}
\begin{itemize}

\item Given an MIU-string $x$, can we construct an ``equivalent'' NT-string $y$?
\label{sec-4-1-1}%
\begin{itemize}

\item If so, we can use number theory to talk about the MIU game.\\
\label{sec-4-1-1-1}%
\item And if so, maybe we can do the same thing to an NT-string?\\
\label{sec-4-1-1-2}%
\item Then we can make NT-theorems talk about NT-theorems.\\
\label{sec-4-1-1-3}%
\item Eventually, we need a theorem to talk about itself. So this is a good first step.\\
\label{sec-4-1-1-4}%
\end{itemize} % ends low level

\item ``Equivalent'' means $y$ is an NT-theorem iff $x$ is an MIU-theorem.
\label{sec-4-1-2}%
\begin{itemize}

\item We want to be able to do this even if we don't know whether $x$ is an MIU-theorem.\\
\label{sec-4-1-2-1}%
\end{itemize} % ends low level

\item It turns out we can.\\
\label{sec-4-1-3}%
\item Won't prove this, but will try to convince.
\label{sec-4-1-4}%
\begin{itemize}

\item Given a number $x$, we construct an NT-statement which is a theorem iff $x$ is the Gödel number of a well-formed MIU-statement.\\
\label{sec-4-1-4-1}%
\item To be precise, iff $x$ contains only the decimal digits 1, 2, 3.\\
\label{sec-4-1-4-2}%
\end{itemize} % ends low level
\end{itemize} % ends low level
\end{frame}
\begin{frame}
\frametitle{Useful tricks}
\label{sec-4-2}
\begin{itemize}

\item Check that a finite set of things each has some property.\\
\label{sec-4-2-1}%
\item Finite sequences are countable.
\label{sec-4-2-2}%
\begin{itemize}

\item Can represent a sequence by a number, and use another number to extract elements from it.\\
\label{sec-4-2-2-1}%
\end{itemize} % ends low level

\item Instead of finding a number that satisfies P, just need to ask ``does $x$ satisfy P?'' Then use ∃.\\
\label{sec-4-2-3}%
\end{itemize} % ends low level
\end{frame}
\begin{frame}
\frametitle{MIU-FORMED}
\label{sec-4-3}
\begin{itemize}

\item $\text{LOG}\{a,b\}$ → $1\shl{b} > a ∧ 1\shl{b-1} ≤ a$\\
\label{sec-4-3-1}%
\item $\text{GOOD}\{a\}$ → $a = 1 ∨ a = 2 ∨ a = 3$\\
\label{sec-4-3-2}%
\item $\text{NTH}\{a,b,c\}$ tests whether the $b$'th digit of $a$ is $c$.\\
\label{sec-4-3-3}%
\item $\text{MIU-FORMED}\{a\}$ →\\
\label{sec-4-3-4}%
\begin{align*}
∃b(&\text{LOG}\{a,b\} ∧\\
   &∃(\text{sequence }x_1\text{ to }x_b) ∀c (c < b ⇒ \text{NTH}\{a,c,x_c\} ∧ \text{GOOD}\{x_c\}))
\end{align*}
\end{itemize} % ends low level
\end{frame}
\begin{frame}
\frametitle{NT-THEOREM}
\label{sec-4-4}
\begin{itemize}

\item If an NT-statement can be Gödel-numberised, so can an NT-derivation.\\
\label{sec-4-4-1}%
\item So given the number of a derivation, and the number of a statement, we can ask\\
\label{sec-4-4-2}%
``Does this derivation prove this statement?''

\item This can be expressed as an NT-statement, $\text{NT-DERIVES}\{a,b\}$.\\
\label{sec-4-4-3}%
\item So we can construct:\\
\label{sec-4-4-4}%
$\text{NT-THEOREM}\{a\}$ → $∃b (\text{NT-DERIVES}\{b,a\})$.

\item Note that constructing a statement is much easier than determining its truth.\\
\label{sec-4-4-5}%
\end{itemize} % ends low level
\end{frame}
\section{``This statement is unprovable''}
\label{sec-5}
\begin{frame}
\frametitle{Self-reference}
\label{sec-5-1}
\begin{itemize}

\item Liar paradox: ``This statement is true.''\\
\label{sec-5-1-1}%
\item Not much better: P - ``Q is true.'' Q - ``P is false.''\\
\label{sec-5-1-2}%
\item Quine:\\
\label{sec-5-1-3}%
``yields falsehood when preceeded by its quotation.'' yields falsehood when preceeded by its quotation.
\begin{itemize}

\item This turns out to be the key.\\
\label{sec-5-1-3-1}%
\end{itemize} % ends low level
\end{itemize} % ends low level
\end{frame}
\begin{frame}
\frametitle{Quining}
\label{sec-5-2}
\begin{itemize}

\item If an NT-statement has free variables, we can substitute any number we like into them.\\
\label{sec-5-2-1}%
\item In particular, we can substitute the Gödel number of the original statement.\\
\label{sec-5-2-2}%
\item Construct a formula $\text{QUINE}\{a,b\}$ which tests whether $b$ is ``$a$ quined''.
\label{sec-5-2-3}%
\begin{itemize}

\item Essentially, $\text{QUINE}\{a,b\}$ → $b = a\{a\}$\\
\label{sec-5-2-3-1}%
\end{itemize} % ends low level
\end{itemize} % ends low level
\end{frame}
\begin{frame}
\frametitle{``This statement is unprovable''}
\label{sec-5-3}
\begin{itemize}

\item Let $U\{a\}$ → $¬∃b (\text{QUINE}\{a,b\} ∧ \text{NT-THEOREM}\{b\})$\\
\label{sec-5-3-1}%
\item Let $G$ be the quinification of $U$.\\
\label{sec-5-3-2}%
\item In other words, find $G$ such that $\text{QUINE}\{U,G\}$ is true.\\
\label{sec-5-3-3}%
\item $U\{a\}$ says ``$a$ quined is not a theorem.'' Equivalently, ``is not provable.''\\
\label{sec-5-3-4}%
\item So $G$ (= $U\{U\}$) says ``$U$ quined is not provable.''\\
\label{sec-5-3-5}%
\item But $G$ \emph{is} $U$ quined.\\
\label{sec-5-3-6}%
\item Thus, $G$ says ``$G$ is not provable.''\\
\label{sec-5-3-7}%
\end{itemize} % ends low level
\end{frame}
\begin{frame}
\frametitle{Aftermath}
\label{sec-5-4}
\begin{itemize}

\item If $G$ is false, we can find a proof of $G$, and number theory is inconsistent.\\
\label{sec-5-4-1}%
\item If $G$ is true, it can't be proved, so number theory is incomplete.\\
\label{sec-5-4-2}%
\item Since $¬G$ says ``$G$ is provable'', $¬G$ asserts its own negation.\\
\label{sec-5-4-3}%
\item So neither $G$ nor $¬G$ is a theorem.\\
\label{sec-5-4-4}%
\end{itemize} % ends low level
\end{frame}

\end{document}
